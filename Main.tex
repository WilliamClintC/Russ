\documentclass[12pt]{article}
\usepackage[right=1.25in,left=1.25in,top=1.1in,bottom=1.1in]{geometry}
\usepackage{hyperref}
\hypersetup{colorlinks, citecolor=blue, filecolor=blue, linkcolor=blue, urlcolor=blue}
\usepackage{graphicx}
\usepackage{url}
\usepackage[round]{natbib}
\usepackage{amsmath,amsthm} 
\usepackage{engord}
\usepackage{float}
\usepackage{subfig}
\usepackage{pdflscape}
\usepackage{booktabs}
\usepackage{pgfplots}
\pgfplotsset{compat=1.14}
\pgfplotsset{every axis label/.append style={font=\tiny}}
\usepackage[labelsep=period]{caption} %% This switches "Table 1: Title" to "Table 1. Title"
\usepackage[T1, T2A]{fontenc}% T2A for Cyrillic font encoding



\usepackage{amssymb} %% Necessary, just for the \checkmark command  in tables.
\usepackage{multirow} %% Necessary if we are doing tables in LaTeX
\usepackage{array}
\usepackage{graphicx}
\usepackage{xr}

\usepackage{setspace}
\onehalfspacing

\usepackage{sectsty}
\sectionfont{\large}
\subsectionfont{\normalsize}
\subsubsectionfont{\normalsize}

\newcommand{\specialcell}[2][c]{\begin{tabular}[#1]{@{}l@{}}#2\end{tabular}}

%%%%%%%%%%%%%%%%%%%%%%%%%%%%%%%%%%%%%%%%%%%%%%%%%%%%%%%%%%%%%

\title{ \vspace*{-2.5cm} \hspace*{-0.5cm}Russian Culture in the Central Bank of Russia}


\author{William\thanks{University of British Columbia.
\href{mailto:TK@TK.edu}{wco@student.ubc.ca}}} 
%\and Author Two\thanks{TK University and NBER.  \href{mailto:TK@TK.edu}{TK@TK.edu}} 
%\and Author Three\thanks{TK University. \href{mailto:TK@TK.edu}{TK@TK.edu}}}

\date{ \vspace*{0.5cm} May, 2023\\
%\textbf{Preliminary and Incomplete. \\ Please do not cite or circulate.}
} 

%%%%%%%%%%%%%%%%%%%%%%%%%%%%%%%%%%%%%%%%%%%%%%%%%%%%%%%%%%%%%

\begin{document}
\nocite{*}
\bgroup
\let\footnoterule\relax

\begin{singlespace}
\maketitle


\begin{abstract}
    \noindent Throughout the years, numerous attempts have been made to damage the Russian Economy and yet Russia's economy is relatively stable. We argue this is the case because of the strong Russian cultural values that inhabit the Central Bank of Russia through its leadership by Elvira Nabiullina. We identify 3 key values, sincerity, loyalty and resiliency, expressed through anecdotes and visually through brooches. 
\end{abstract}
\end{singlespace}
\thispagestyle{empty}

\clearpage
\egroup
\setcounter{page}{1}

%% Temporary tool to track how this paper is structured. Feel free to comment in or out. 
% \tableofcontents
% \bigskip

%%%%%%%%%%%%%%%%%%%%%%%%%%%%%%%%%%%%%%%%%%%%%%%%%%%%%%%%%%%%%
%%%%\section{Introduction\label{sec:introduction}}

\section{Introduction 
\label{sec:Introduction}}

Sanctions have commonly been an avenue to inflict damage to a country without physical military force and without a doubt sanction are powerful tools. Countries like Afghanistan, Iran, Syria, Cuba and Venezuela know these all the well. Against US sanctions, said country’s economies collapse under economic pressures. However, there is an exception. Russia has experienced numerous sanctions over the years. Most notably, all G-7 countries froze Russia’s foreign currency reserve assets. To top it off, Switzerland broke its long history of neutrality to join on the sanctions. History has never seen such a coordinated effort to damage a country’s banking system and yet against all these sanctions Russia’s economy chugs along. In a bout of irony in 2023, we saw numerous bank failures, most notably the US’s Silicon Valley Bank and Switzerland’s Credit Suisse, and none from Russia. 

Such a feat is arguably attributed to Elvira Nabiullina, governor of the Bank of Russia. Her decisive actions gained the respect of central bankers around the world, winning recognition from Euromoney and the Banker as “Central Banker of the Year”. What makes her accomplishments even more impressive is the fact that she is a woman and Muslim ethnic minority, in a male dominated Russian Christian-centric country, being the first and only female head central banker throughout the G8’s history.  We investigate Elvira’s success in the context of the Russian Cultural values that shape her. Through the success of the Central Bank of Russia, Elvira personifies the Russian cultural values of sincerity, loyalty and resiliency.

\section{Sincerity 
\label{sec:Sincerity}}
Elvira is the quintessential Nekrasov woman, a woman that “"will stop a horse at a gallop, enter a burning hut". She has good company, joining the likes of Catherine the great or Pushkin’s Vasilissa Yegorovna. 

\input{fig_tex/Nekrasov}

She embodies Russian sincerity. Today, central bankers follow the unwritten rule of being as dry and emotionless as possible, a heard learned lesson from the “2013 Taper Tantrum”, wherein just the mere hint of reduced asset purchases, triggered a massive market sell off. Elvira is the exception, visually broadcasting her thoughts and insecurities in the form of her brooches, mirroring the contrast between the Russian culture of sincerity against western practices of insincere politeness. 

This is not to be confused with incompetency, but more of adaptability. Elvira dawns western influences. She wore a dove as she lowered interest rates, a hawk pin as she raised interest rates. Doves and hawks being common metaphors for central banking practices. 
\input{fig_tex/Dove and Hawk}

She even joined on the US central bank’s prayers for a V-shaped pandemic recovery in her V- shaped brooch. A rain cloud when dampening inflation expectations, 
\input{fig_tex/V and Cloud}

a dot when ending the cycle of monetary easing,the golden spiral when containing spiraling inflation,
\input{fig_tex/dot and spiral}
scales when balancing monetary policy, pause button when keeping rates the same, the list goes on. 
\input{fig_tex/scale and pause}


In adopting western influences, she cultures them with Russian influences. In February 2020, a new virus was getting a foothold in the economy causing economic uncertainty, when asked if she was going to be a dove or a hawk, Elvira replied she was a stork with a stork brooch to match, symbolizing Russia’s deviation from global central banking trends. \input{fig_tex/stork} Wherein at the time, the common central bank were blindly being pulled to follow the economic might of the US's central bank, Elvira adapted her policies for the Russian economy.  



\section{Loyalty to Motherland
\label{sec:Loyalty}}
Elvira’s attentiveness to the local economy exemplifies the Russian cultural value of loyalty to homeland akin to Pushkin’s Pyotr Grinyov or the tale of the Teremok. Whenever setbacks arise, practicality takes over. Blaming and grudges are seconded to rebuilding and improving. In 2022, more than 80\% of Russia’s foreign reserves was seized. The Russian ruble plunged by more than 25\%, inflation was rising, and the interest rate doubled to 20\%. She mourned the looming collapse of the economy, wearing no brooches but a black funeral attire.
\input{fig_tex/BlackAttire}
It was a dire situation that required major policy action.  She pondered her choices, join the mass migration of Russian brain drain or stay, potentially making enemies and losing credibility as she worked to stabilize the economy. Elvira once stated, if capital controls were ever introduced I would leave and yet she implements the capital controls herself, forcing Russians to sell 80\% of their foreign currency while banning foreign currency transfers. Throughout her career, she was also responsible for the publicly unpopular act of “bailing out” Russia’s top banks and oil companies.  During times of financial crisis, she fought against the short-term pressures to print more money, which hurt and disgruntled everyday Russians. Her bailouts guaranteed stability, while her thrift capped hyperinflation.  Against this backdrop President Putin once said “I know how unhappy the real sector of the economy is with the raised interest rates (not printing money), but we could end up like Turkey if it isn’t done”, referring to Turkey’s struggle from hyperinflation. All this shows is Elvira’s loyalty to her homeland and the lengths that she will go to protect it. Safe to say Elvira is not an outlier, all throughout her time in office the central bank had a turnover rate of only 1 in 1000. 

\section{Resiliency 
\label{sec:Resiliency}}
Such drastic actions exemplify Russian resiliency against extremes. Just like Dovlatov's cousin, Russia excels in the worst situations and crumples otherwise, forming waves. Waves of stability and upheaval, abundance and famine, harsh cold and heat. These are exemplified in events such as the Mongol invasion, Russian civil war, Lenin’s red terror or the reign of Ivan the terrible. Climatologists are well aware that Russia’s land consistently experience the greatest temperature ranges on Earth. A more recent example would be the waves of Covid-19 outbreaks or unsteady oil price’s, in which the Russian economy is dependent on. Elvira nods to this past in her speech where she wore a wave brooch. Against these waves Elvira’s black and red bird brooch suggests, Russia is not a dove or a hawk but a bull finch, a small bird in Russia that withstands the harshest winters. In a play of words bull is a finance metaphor for a strong market, suggesting Russia’s resiliency and competency. 
\input{fig_tex/Wave and Bullfinch}
2014 marked Elvira’s new role as the head of the Central Bank of Russia (CBR). As history proves, a crisis immediately stepped on to her door with a big collapse in oil prices. The confidence in  the rouble quickly faltered leading to hyperinflation. The standard playbook at this time was to project confidence by putting out reassuring statements, fixing the exchange rate and keeping interest rates low. Of course, this strategy hinged on oil prices eventually recovering. No one expected the new head of the CBR to do any different. Against the conservative establishment, her ideas and influence were questioned. It was clear that they preferred the more standard pick, neo-communist economist, Sergei Glazyev. Her handling of the situation would mark the fruitful start or short-lived end of her career. Against these pressures, Elvira acted.

She drew on her economic knowledge, on how the fixed exchange rates facilitated the 1992 sterling  crisis and on how dependency on high oil prices and low interest rates in Venezuela facilitated hyper inflation and economic collapse. Going against the standard playbook, she exposed Russia’s rouble to the public, floating the exchange rate and increasing interest rates to an all time high of 20\%, higher than the infamous 1980 US Paul Volcker increase. Floating the rouble, exposed the weakness in the Russian currency with the exchange rate falling sharply. The sharp Volcker like rate increase threatened to destroy the Russian economy already hurting from low oil prices. 

In the end Elvira’s battle with hyperinflation was a success. Oil prices never recovered while more and more sanctions against Russia were implemented and yet inflation went to a record low of 2.2\%. The standard playbook that hinged on the recovery of oil prices would have certainly destroyed the Russian economy. History has shown her decisive actions was exactly what needed to be done.

Against this feat, Elvira dawned a Nevalyashka Doll, a Russian wobble toy that always stands when pushed, signifying the Russian unwavering spirit. Breaking the national peg (fixed exchange rate) more often is seen as a sign of currency weakness. In Nevalyashka Doll fashion, the break became a sign of competency and resilience.  
\input{fig_tex/Doll}


\section{Conclusion 
\label{sec:Conclusion}}

Despite previous successes Russians know too well the tragedy of the Kolobok, complacency is a tragedy. Today there is a new crisis, the ruble has lost half of its value and inflation has jumped to 20\%. Weather Russian resiliency shines or joins the troughs of the wave, time will tell. But what is clear is that the Russian economy is driven by a rich culture capable of handling the worst setbacks. As times got tough, we saw Elvira sport more aggressive brooches like a jaguar brooch and a bow and arrow brooch, against the world who expected a white flag brooch. 
\input{fig_tex/jaguar and arrow}
\input{fig_tex/Hero}








%%%%%%%%%%%%%%%%%%%%%%%%%%%%%%%%%%%%%%%%%%%%%%%%%
\clearpage
\begin{singlespace}
%\bibliographystyle{plainnat}
%\bibliographystyle{chicago}
\bibliographystyle{aer}
\bibliography{our-cites.bib}
\end{singlespace}
%%%%%%%%%%%%%%%%%%%%%%%%%%%%%%%%%%%%%%%%%%%%%%%%%


%%%%%%%%%%%%%%%%%%%%%%%%%%%%%%%%%%%%%%%%%%%%%%%%%
%%%%% These commands start the appendix and change the Table & Figure numbering
\newpage
\appendix
\setcounter{table}{0}
\renewcommand{\tablename}{Appendix Table}
\renewcommand{\figurename}{Appendix Figure}
\renewcommand{\thetable}{A\arabic{table}}
\setcounter{figure}{0}
\renewcommand{\thefigure}{A\arabic{figure}}
%%%%%%%%%%%%%%%%%%%%%%%%%%%%%%%%%%%%%%%%%%%%%%%%%


\end{document}
